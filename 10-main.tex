
\documentclass{LU}

\title{Digitālās aparatūras projektēšanas tiešsaistes platforma}	   
\thesistype{Bakalaura darbs}
\author{Krišjānis Veinbahs}
\studentid{kv18042}
\supervisor{prof., Dr. Dat. Leo Seļāvo}
\university{Latvijas Universitāte}
\faculty{Datorikas Fakultāte}
\location{Rīga}


% apzīmējumu saraksts
\makeglossaries

\newglossaryentry{example}
{
    name=FPGA,
    description={Lauka programmējamās ventiļu matricas}
}

\begin{document}

\maketitle

\renewcommand{\abstractname}{}
\begin{abstract}
    \begin{center}
    \Large\textbf{ANOTĀCIJA}\\
    \end{center}
    \vspace{1.5\baselineskip}

    Darba mērķis ir izstrādāt un analizēt tiešsaistes platformu digitālās aparatūras attīstītājrīku
    attālinātai pārvaldībai, programmaparatūras jeb projektējumu glabāšanai un šīs programmaparatūras testēšanai
    platformā pieejamajos attīstītājrīkos.

    Darbā tiek apskatīta platformas darbības mehānismu modelēšana, sarežģījumi šādas
    platformas izstrādē, izmantotie risinājumi un to arhitektūra.

    Precīzāk, darbā tiek apskatīta vāja reāllaika attālinātas mijiedarbības
    realizācija starp platformas lietotājiem un digitālo aparatūru, tai skaitā attālināta
    noprogrammēšana, testēšana un izmantošana. Tiek apskatītas, analizētas un
    izstrādātas arī dažādas papildus funkcionalitātes, kas uzlabo šo platformas pieredzi.

    \textbf{Atslēgas vārdi}: FPGA, UART, reāllaiks, notikumu sistēma, aktieru modelis.
\end{abstract}
 

\selectlanguage{english}
\renewcommand{\abstractname}{}
\begin{abstract}
    \begin{center}
    \Large\textbf{ABSTRACT}\\
    \Large\text{AN ONLINE PLATFORM FOR HARDWARE}\\
    \Large\text{DESIGN, ENGINEERING AND TESTING}\\
    \end{center}
    \vspace{1.5\baselineskip}

    This work contains details regarding the development and analysis of an online platform for
    managing digital hardware development boards and for managing and testing prototype firmware on these boards.

    Also this work encompasses process modelling of such a platform, complications encountered in development and
    the solutions and architecture for solving these complications.  

    More specifically this work describes a soft realtime system for remote interaction between platform users and 
    available digital hardware development boards i.e. board programming, prototype firmware testing and information exchange.
    In addition some extra features for better user experience in the platform are implemented and described.  

    \textbf{Keywords}: FPGA, UART, realtime, event system, actor model.
\end{abstract}
\selectlanguage{latvian}

\pagenumbering{gobble} 

\tableofcontents

%------------------------------------------------APZĪMĒJUMI---------------------------------------------------------

\printglossary[type=main,title={APZĪMĒJUMU SARAKSTS},toctitle={Apzīmējumu saraksts}]

\pagenumbering{arabic} % sākam numurēt lapas no apzīmējumu saraksta (3. pielikums iekš LU 03.02.2012, 1/38 ) 

\chapter*{Ievads} % * nepieliks numuru pie nosaukuma
\addcontentsline{toc}{chapter}{Ievads}
\pagestyle{plain}
Digitālās aparatūras projektēšana ir loģisku elementu konfigurāciju un savstarpējo savienojumu plānošana, lai
rezultātā iegūtu jeb projektētu ierīci jeb aparatūru, kas veiktu autora jeb izstrādātāja iedomāto funkcionalitāti, piemēram,
lai izveidotu veļasmašīnas kontrolieri vai datora procesoru, vai citu digitālu risinājumu.  
  
Digitālo aparatūru jeb tās loģisko elementu konfigurāciju var fiziski ražot silikona platēs, tādējādi iegūstot patstāvīgu integrālo
shēmu, vai, alternatīvi, ir iespējams šo loģisko elementu konfigurāciju augšupielādēt attīstītājrīkā, tam nolasot šo konfigurāciju un 
pārkonfigurējot sevī esošu FPGA mikroshēmu, lai tā darbotos atbilstoši dotajai konfigurācijai.  

Darba ietvaros uzmanība tiek koncentrēta uz šī digitālās aparatūras izstrādes dzīvescikla prototipēšanas fāzi jeb brīdi, kad
tiek izstrādāts programmaparatūras prototips, tas tiek augšupielādēts attīstītājrīkā un tiek pārbaudīta attīstītājrīka uzvedība
atbilstoši vēlemajai gala aparatūras funkcionalitātei.

Šī darba mērķis ir mēģināt uzlabot šo izstrādes procesu pārveidojot attīstītājrīka jeb fiziskās tehnikas pārvaldību no fiziska procesa
par digitālu, kā arī



%------------------------------------------------DARBS--------------------------------------------------------------

\chapter{Teorija}
\section{Problēmas pamatnostādne}

Situācija, problēma

Esošie risinājumi īsumā (teikums ar referencēm)

"However..." - ko tie neatrisina vai atrisina nepilnīgi

"Šajā darbā tas tiek atrisināts šādi..." (īsumā, detaļas būs vēlākajās nodaļās)

Standarta paragrāfs: sekojošā nodaļā stāstīts par to, nākamajā par to, ...


Šajā nodaļa mēs apskatam pētāmās problēmas teorētisko pusi.
Veicam daudz $copy + paste + alter \vee trnslate$.
Galvenais ir salikt labi daudz atsauču \cite{kant2018recent}.
Patiesībā katru teikumu varam uztver kā BS un atsauces ir veids kā pateikt, ka tas nav \gls{example}.
\glslink{example}{ekzamplis}.

\section{Saistība ar citiem pētījumiem}

Esošo risinājumu apskats - pieminam kursa darbā rakstīto.

Definē problēmas apgabalu un izmēru un saistību ar citiem līdzīgiem darbiem.  


\chapter{Risinājums}
\section{Motivācija}
Kādēļ es taisīju to, ko es taisīju?  

Kā tas ir labāk par to, kas eksistē tagad?  

\section{Risinājums}

Ko tad es īsti uztaisīju - lielos un mazos vilcienos?
  
No kādām detaļām sastāv mans risinājums?

Šeit mēs aprakstam pāris rindkopās to, kamā mēs izpletīsimies nākamajās nodaļās

Aptuvena ideja: Es uztaisīju aktieru modelī balstītu sistēmu, kas nodrošina centralizētu klientu un aģentu pārvaldību
ar pieņēmumu, ka gan klienti, gan aģenti jebkurā brīdī varētu atslēgties no sistēmas. Aktieru modelis nosaka, ka sistēma
ir arī notikumu sistēma, tātad visas izmaiņas sistēmā tiek padotas apkārt izmantojot ziņas jeb vēstules. Notikumu sistēma
nozīmē, ka ir ļoti viegli klientu un aģentu darbībai paralēli arī žurnalēt to stāvokli piem. datubāzē. Papildus šeit jāpiemin
kādi tad ir tie aktieri, kas tiek izmantoti šajā sistēmā. 

Jāpiemin, ko tad tā sistēma īsti nodrošina - ar Scala Play Framework un HTTP REST un Twirl un Slick tā nodrošina biznesa datu CRUD pārvaldību un
basic auth autentifikāciju, savukārt, ar Scala Play Framework un WebSocket un JSON un aktieriem tā nodrošina vāja reāllaika starpkomunikāciju.  

Vēl jāpiemin, kāda jēga no šīs CRUD pārvaldības, autentifikācijas un vāja reāllaika komunikāciju. CRUD mums ļauj pārvaldīt lietotājus, dzelžus,
programmaparatūru. Vājā reāllaika komunikācija ļauj augšupielādēt mums programmatūru dzelžos, tad sūtīt un saņemt informāciju starp klientu un dzelzi, 
kurā darbinās programmaparatūra, tātad ļauj mums mijiedarboties ar dzelzi. Autentifikācija ļauj pārvaldīt piekļuvi CRUD datiem un
komunikācijai ar dzelžiem.

Vēl vajadzētu pieminēt, neskaitot iepriekšminētās salīdzinoši tehniskās detaļas, kā šī sistēma atrisina biznesa problēmu jeb izstrādes procesa
digitalizāciju. Kā ar šo sistēmu var iesūtīt programmaparatūru, dabūt to dzelzī, mijiedarboties ar programmaparatūru, kas darbinās dzelzī, izmantojot
"virtuālās saskarnes".  

\subsection{DIP platforma}

Šeit varbūt varētu pārkopēt "Piedāvātais risinājums" saturu, lai būtu viena skaidra, īsa, kodolīga sekcija par platformu. 

\subsection{Komunikācijas protokoli}

1. HTTP REST - Biznesa entītiju pārvaldība centrālajā serverī
2. HTTP WebSocket - Vāja reāllaika komunikācija starp centrālo serveri un klientiem/aģentiem
3. Akka - Centrālā servera aktieru vāja reāllaika komunikācija
4. Basic auth un session cookies - autentifikācija iepriekšminētajiem (var pieminēt, var nepieminēt)
5. USB serial - Izmantots komunikācijai starp aģentu un dzelzi

\subsection{Notikumu sistēmas algoritma realizācija}

Notikumu dzinējs ar blakusefektiem - notikumu sistēmas realizācija gan klientā, gan aģentā, gan serverī.

Actor model (every actor defines its inputs)

Message passing.

Event engines.

Event sourcing.

Purity and side-effects.

\subsection{Aģenti un klienti un to starpkomunikācija}

Kā strādā aģenti, kā tie klausās komandas no platformas, kā tie veic programmēšanu,
kā tie uztur virtuālās saskarnes komunikācijas?

Kā strādā klienti? Kā tie sūta CRUD izmaiņas? Kā tie sūta aģentiem komandas? Kā tie
komunicē ar aģentu programmaparatūru izmantojot seriālo portu.

UART, Virtual Interfaces (webcam, serial connection)

\subsection{Virtuālā saskarne "tīmekļkamera"}

Visi veidi kā es mēģināju realizēt tīmekļkameras virtuālo saskarni.

Cik labi tas strādā?

Kā to varētu izdarīt vēl labāk?

Īsumā - OGG video streaming

\subsection{Virtuālās saskarnes "hexbytes" un "buttonleds"}

Apraksts dažādu veidu 1-pret-1 baitu ziņapmaiņas virtuālajām saskarnēm.

Ko ar tām var panākt?

Ko ar tām nevar panākt?

\subsection{Virtuālā saskarne "MinOS"}

Apraksts chunk-to-chunk jeb uz seriālā porta paketēm balstītu ziņapmaiņas saskarni.

Pielikums teksts: Protokola sintakse https://github.com/kshaa/dip-testbed-dist/blob/master/prototypes/06-anvyl-min-os/syntax.bnf  

Pielikums attēls: TUI saskarne https://github.com/kshaa/dip-testbed-dist/blob/master/docs/assets/UHuU1Ur8e0CgoTmsm5khLuOJH.png

Pielikums attēls: 

Ko ar šo saskarni var panākt?

Ko ar šo saskarni nevar panākt?

\subsection{Infrastruktūras pārvaldība}

Kā laboratorijas īpašnieki pieslēdz savus dzelžus platformai?

Kā klienti pieslēdzas sistēmai un gūst iespēju izmantot dzelžus?

Kā izstrādātājs jeb darba autors jeb es automatizē versionēšanu, artefaktu pārvaldību?

Docker cross-platform w/ buildx i.e. buildkit

Board management w/ Ansible

CICD deployment procesi

\subsection{Testēšana un uzturēšana}

Šo es praktiski neesmu paspējis izdarīt, bet šis būtu interesanti:

Waveform recordings

WASM testing

Heartbeats

\subsection{Potenciāli uzlabojumi}

Improvements using Scala.js and Scala Native (client, agent, web)?

Improvements w/ WASM virtual interfaces?


\chapter{Rezultāti}
\section{Laboratorijas uzstādīšana}

Īss apraksts kā uzstādīt laboratoriju jeb dzelžus platformā.

\section{Veļasmašīnas izstrāde platformā}

Īss apraksts kā projektēt veļasmašīnu un prototipēt to platformā.

\section{Datu apstrāde platformā}

Īss apraksts kā apstrādāt datus platformā.  

Ideja par to, ka var iesūtīt MinOS teksta paketi, saņemt atbildi JSON formātā.  

Jeb īsumā apraksts par to, ka var izmantot FPGA as a service iekš CLI.  

Ar domu: dipclient hardware-serial-monitor -t minosrequest -r "potat" -j true

\section{Veiktspēja}

Varētu aši uzcept kkādus benchmark ar to minosrequest un saprast, cik ātri var saņemt atbildes, cik ātri notiek apstrāde un tā.


%----------------------------------------------SECINĀJUMI----------------------------------------------------------

\chapter*{Secinājumi}
\addcontentsline{toc}{chapter}{Secinājumi}
Pieminēt, ka uztaisīju platformu ar tīmekļa serveri.

Pieminēt, ka uztaisīju klientus un aģentus.

Pieminēt, ka uzprojektēju MinOS Verilogā.

Pieminēt, ka mērķis, manuprāt, izpildīts, jo var digitāli pārvaldīt dzelžus un piekļuvi tiem, var tos digitāli programmēt un testēt.   


%---------------------------------------------LIETERATŪRA----------------------------------------------------------
\renewcommand{\bibname}{Izmantotā literatūra un avoti}
\bibliographystyle{abbrv}
\bibliography{80-main}
\addcontentsline{toc}{chapter}{Izmantotā literatūra un avoti}


%----------------------------------------------PIELIKUMS----------------------------------------------------------

\begin{appendices}
\chapter*{Pielikums}
\renewcommand{\thesection}{\arabic{section}}
\titleformat{\section}{\normalfont\large\bfseries}{\thesection. pielikums.}{1em}{}

\section{skaists pielikums, lai vairāk lapu}

;)
\end{appendices}

%---------------------------------------------REĢISTRĀCIJAS LAPA (TODO)-------------------------------------------

\chapter*{Reģistrācijas lapa}
Bakalaura darbs "Digitālās aparatūras projektēšanas tiešsaistes platforma" izstrādāts LU Datorikas fakultātē.
\vspace{1.5\baselineskip}

Ar savu parakstu apliecinu, ka pētījums veikts patstāvīgi, izmantoti tikai tajā norādītie informācijas avoti un
iesniegtā darba elektroniskā kopija atbilst izdrukai.

Darba autors: \makebox[1.5in]{\hrulefill} Krišjānis Veinbahs
\vspace{1.5\baselineskip}

Rekomendēju/nerekomendēju darbu aizstāvēšanai (nederīgo svītro vadītājs)

Darba vadītājs: prof., Dr. Dat. Leo Seļāvo \makebox[1.5in]{\hrulefill} \makebox[.25in]{\hrulefill}.06.2022.
\vspace{1.5\baselineskip}

Darbs iesniegts Datorikas fakultātē \makebox[1.5in]{\hrulefill} \makebox[.25in]{\hrulefill}.06.2022.
\medskip

Dekāna pilnvarotā persona: vecākā metodiķe Ārija Sproģe \makebox[1.5in]{\hrulefill}
\vspace{1.5\baselineskip}

Darbs aizstāvēts bakalaura gala pārbaudījuma komisijas sēdē

\makebox[.25in]{\hrulefill}.06.2022. prot. Nr. \makebox[.25in]{\hrulefill}
\vspace{1.5\baselineskip}

Komisijas sekretārs(-e):



\end{document}
