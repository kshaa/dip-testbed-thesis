\renewcommand{\thesection}{\arabic{section}}
\titleformat{\section}{\normalfont\large\bfseries}{\thesection. Pielikums.}{1em}{}

\section{Digitālās aparatūra "pieskaitītājs un atņēmējs"}
\label{att:counter}

Sekojošais kods ir sarakstīts Verilog valodā un to sintezējot ir iespējams iegūt
loģisko elementu konfigurāciju, ko, savukārt, jau ir iespējams augšupielādēt
kādā digitālās aparatūras attīstītājrīkā.

\begin{lstlisting}
    // counter.v
    module main(
        input BTN0,
        input BTN1,
        output LD0,
        output LD1,
        output LD2,
        output LD3,
        output LD4,
        output LD5,
        output LD6,
        output LD7
    );
        // Counter instance
        reg [7:0] counter;
        wire BTN = BTN0 || BTN1;
        always @(posedge BTN)
        begin
            if (BTN0) begin
                counter <= counter - 1;
            end else if (BTN1) begin
                counter <= counter + 1;
            end
        end
    
        // Assign counter value to physical LEDs
        assign LD0 = counter[0];
        assign LD1 = counter[1];
        assign LD2 = counter[2];
        assign LD3 = counter[3];
        assign LD4 = counter[4];
        assign LD5 = counter[5];
        assign LD6 = counter[6];
        assign LD7 = counter[7];
    endmodule    
\end{lstlisting}
  