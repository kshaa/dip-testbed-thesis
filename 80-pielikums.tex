\renewcommand{\thesection}{\arabic{section}}
\titleformat{\section}{\normalfont\large\bfseries}{\thesection. Pielikums.}{1em}{}

\section{Digitālās aparatūra "pieskaitītājs un atņēmējs"}
\label{att:counter}

Lai ilustrētu digitālās projektēšanas procesu, tika Verilog \gls{hdl} valodā
projektēta digitāla iekārta, kas, izmantojot spiedpogas, skaitlim pieskaita un
atņem skaitli viens.

Sintezējot šo \glslink{firmware}{programmaparatūru} ir iespējams iegūt loģisko
elementu konfigurāciju, ko, savukārt, jau ir iespējams augšupielādēt kādā
digitālās \glslink{board}{aparatūras} attīstītājrīkā, lai testētu
\glslink{board}{aparatūras} funkcionalitāti. 

Kods satur spiedpogu signālus \lstinline!BTN0-1! kā ievaddatus un LED gaismas
\lstinline!LD0-7! kā izvaddatus. \glslink{firmware}{Programmaparatūra} spēj
skaitīt no \lstinline!0! līdz \lstinline!255! jeb \(2^8\) un izvada šī skaitļa
vērtību kā 8 bitus LED gaismu signālos.

\begin{lstlisting}
// counter.v
module main(
    input BTN0,
    input BTN1,
    output LD0,
    output LD1,
    output LD2,
    output LD3,
    output LD4,
    output LD5,
    output LD6,
    output LD7
);
    // Counter instance
    reg [7:0] counter;
    wire BTN = BTN0 || BTN1;
    always @(posedge BTN)
    begin
        if (BTN0) begin
            counter <= counter - 1;
        end else if (BTN1) begin
            counter <= counter + 1;
        end
    end

    // Assign counter value to physical LEDs
    assign LD0 = counter[0];
    assign LD1 = counter[1];
    assign LD2 = counter[2];
    assign LD3 = counter[3];
    assign LD4 = counter[4];
    assign LD5 = counter[5];
    assign LD6 = counter[6];
    assign LD7 = counter[7];
endmodule    
\end{lstlisting}
  
\section{Multifunkcionālās saskarnes "MinOS" pakotņu protokola BNF sintakse}
\label{att:minosbnf}

Lai nodrošinātu multifunkcionālajā, vairāku grafisko elementu virtuālajā
\glslink{vinterface}{saskarnē} "MinOS" komunikāciju starp digitālo iekārtu un
lietotāja komandu rindas klientu jeb virtuālo \glslink{vinterface}{saskarni}, ir
nepieciešams datu apmaiņas protokols, kas ļautu sūtīt vairākus baitus tīklā un
interpretēt tos kā paketes ar tipu, sākumu, beigām un saturu.

Sekojošais saturs ir BNF formātā aprakstīta sintakse protokolam, kas tiek
izmantots, lai baitu straumē iekodētu vairāku baitu paketes. Šis protokols tiek
izmantots, lai realizētu MinOS virtuālās \glslink{vinterface}{saskarnes}
protokolu. Simbols \lstinline!<chunk>! satur pilnu protokola ziņas sintaksi. 

\begin{lstlisting}
<hex_digit_null>     ::= "0"
<hex_digit_non_null> ::= [1-9]
<hex_digit_one>      ::= "1"
<hex_digit_other>    ::= [2-9]
<hex_digit_any>      ::= [0-9]
<hex_letter>         ::= [A-F]

<hex_symbol_null>     ::= <hex_digit_null>
<hex_symbol_non_null> ::= <hex_digit_non_null> | <hex_letter>
<hex_symbol_one>      ::= <hex_digit_one>
<hex_symbol_other>    ::= <hex_digit_other> | <hex_letter>
<hex_symbol_any>      ::= <hex_digit_any> | <hex_letter>

<nibble_null>      ::= <hex_symbol_null>
<nibble_non_null>  ::= <hex_symbol_non_null>
<nibble_one>       ::= <hex_symbol_one>
<nibble_other>     ::= <hex_symbol_other>
<nibble_any>       ::= <hex_symbol_any>

<byte_prefix> ::= "0x"
<byte_suffix> ::= " "

<byte_null>       ::= 
    <byte_prefix> <nibble_null> <nibble_null> <byte_suffix>
<byte_non_null_a> ::= 
    <byte_prefix> <nibble_non_null> <nibble_any> <byte_suffix>
<byte_non_null_b> ::= 
    <byte_prefix> <nibble_null> <nibble_non_null> <byte_suffix>
<byte_non_null>   ::= 
    <byte_non_null_a> | <byte_non_null_b>
<byte_one>        ::= 
    <byte_prefix> <nibble_null> <nibble_one> <byte_suffix>
<byte_other_a>    ::= 
    <byte_prefix> <nibble_non_null> <nibble_any> <byte_suffix>
<byte_other_b>    ::= 
    <byte_prefix> <nibble_null> <nibble_other> <byte_suffix>
<byte_other>      ::= 
    <byte_other_a> | <byte_other_b>

<unescaped_symbol>     ::= <byte_non_null>
<escaped_symbol_null>  ::= <byte_null> <byte_null>
<escaped_symbol_one>   ::= <byte_null> <byte_one>
<escaped_symbol_other> ::= <byte_null> <byte_other>

<chunk_type>  ::= <escaped_symbol_other>
<chunk_start> ::= <chunk_type>

<content_symbol> ::= <unescaped_symbol> | <escaped_symbol_null>
<chunk_content>  ::= <content_symbol>*

<chunk_end> ::= <escaped_symbol_one>

<chunk> ::= <chunk_start> <chunk_content> <chunk_end>
\end{lstlisting}

\section{Multifunkcionālās saskarnes "MinOS" saskarnes ziņu ieraksts}
\label{att:minosrequest}

Lai testētu digitālo aparatūru nevis manuāli grafiskā saskarnē, bet gan
programmatiski, tika realizēta iespēja definēt JSON formāta teksta failu ar
\glslink{vinterface}{saskarnē} sūtāmajām baitu paketes. Izmantojot šo "MinOS"
programmatisko modeli, ir iespējams nosūtīt iepriekšdefinētas
\glslink{vinterface}{saskarnes} paketes un atpakaļ saņemt jaunu JSON formāta
teksta failu, kas satur aparatūras atpakaļ atsūtītās baitu paketes. Šo
\glslink{vinterface}{saskarnes} izdoto saņemto pakešu JSON teksta failu jau
savukārt var programmatiski apstrādāt ar jebkuru programmēšanas valodu, lai
veiktu jebkāda veida testus jeb pārbaudes.

Sekojošais pielikums ir daļa no šīs programmatiskās \lstinline!minosrequest!
virtuālās \glslink{vinterface}{saskarnes} atgrieztā JSON formāta teksta faila.
Ierakstā redzams tā 4.5 sekunžu izpildes laiks \lstinline!tresholdTime!,
maksimālo saņemto ziņu skaits \lstinline!-1! jeb neierobežots skaits, sākuma
laika zīmogs \lstinline!startTimeStamp!, saņemta displeja pikseļa ziņa
\lstinline!display!, kurā 63. pikselis mainīja savus krāsas datus uz
\lstinline!0x00! laikā \lstinline!sentTimestamp!.

\begin{lstlisting}[caption={\lstinline!minosrequest! \glslink{firmware}{programmaparatūras} ziņu ieraksts JSON formātā},label={lst:buttonleds},captionpos=b]
    [...]
    {
        "type": "display",
        "payload": {
          "index": 63,
          "bits": [
            false,
            false,
            false,
            false,
            false,
            false,
            false,
            false
          ]
        },
        "sentAt": 4490.699462890625,
        "sentTimestamp": "2022-04-04 20:49:37.879356",
        "outgoing": false
      }
    ],
    "tresholdTime": 4500,
    "tresholdChunks": -1,
    "startTime": 1649094573388.6555,
    "startTimeStamp": "2022-04-04 20:49:33.388660"
  }
}
\end{lstlisting}
