\renewcommand{\thesection}{\arabic{section}}
\titleformat{\section}{\normalfont\large\bfseries}{\thesection. Pielikums.}{1em}{}

\section{Digitālās aparatūra "pieskaitītājs un atņēmējs"}
\label{att:counter}

Sekojošais kods ir sarakstīts Verilog valodā un to sintezējot ir iespējams iegūt
loģisko elementu konfigurāciju, ko, savukārt, jau ir iespējams augšupielādēt
kādā digitālās aparatūras attīstītājrīkā.

\begin{lstlisting}
// counter.v
module main(
    input BTN0,
    input BTN1,
    output LD0,
    output LD1,
    output LD2,
    output LD3,
    output LD4,
    output LD5,
    output LD6,
    output LD7
);
    // Counter instance
    reg [7:0] counter;
    wire BTN = BTN0 || BTN1;
    always @(posedge BTN)
    begin
        if (BTN0) begin
            counter <= counter - 1;
        end else if (BTN1) begin
            counter <= counter + 1;
        end
    end

    // Assign counter value to physical LEDs
    assign LD0 = counter[0];
    assign LD1 = counter[1];
    assign LD2 = counter[2];
    assign LD3 = counter[3];
    assign LD4 = counter[4];
    assign LD5 = counter[5];
    assign LD6 = counter[6];
    assign LD7 = counter[7];
endmodule    
\end{lstlisting}
  
\section{MinOS pakotņu protokola BNF sintakse}
\label{att:minosbnf}

Sekojošais saturs ir BNF formātā aprakstīta sintakse protokolam, kas tiek
izmantots, lai baitu straumē iekodētu vairāku baitu paketes. Šis protokols tiek
izmantots, lai realizētu MinOS virtuālās saskarnes protokolu.

\begin{lstlisting}
<hex_digit_null>     ::= "0"
<hex_digit_non_null> ::= [1-9]
<hex_digit_one>      ::= "1"
<hex_digit_other>    ::= [2-9]
<hex_digit_any>      ::= [0-9]
<hex_letter>         ::= [A-F]

<hex_symbol_null>     ::= <hex_digit_null>
<hex_symbol_non_null> ::= <hex_digit_non_null> | <hex_letter>
<hex_symbol_one>      ::= <hex_digit_one>
<hex_symbol_other>    ::= <hex_digit_other> | <hex_letter>
<hex_symbol_any>      ::= <hex_digit_any> | <hex_letter>

<nibble_null>      ::= <hex_symbol_null>
<nibble_non_null>  ::= <hex_symbol_non_null>
<nibble_one>       ::= <hex_symbol_one>
<nibble_other>     ::= <hex_symbol_other>
<nibble_any>       ::= <hex_symbol_any>

<byte_prefix> ::= "0x"
<byte_suffix> ::= " "

<byte_null>       ::= 
    <byte_prefix> <nibble_null> <nibble_null> <byte_suffix>
<byte_non_null_a> ::= 
    <byte_prefix> <nibble_non_null> <nibble_any> <byte_suffix>
<byte_non_null_b> ::= 
    <byte_prefix> <nibble_null> <nibble_non_null> <byte_suffix>
<byte_non_null>   ::= 
    <byte_non_null_a> | <byte_non_null_b>
<byte_one>        ::= 
    <byte_prefix> <nibble_null> <nibble_one> <byte_suffix>
<byte_other_a>    ::= 
    <byte_prefix> <nibble_non_null> <nibble_any> <byte_suffix>
<byte_other_b>    ::= 
    <byte_prefix> <nibble_null> <nibble_other> <byte_suffix>
<byte_other>      ::= 
    <byte_other_a> | <byte_other_b>

<unescaped_symbol>     ::= <byte_non_null>
<escaped_symbol_null>  ::= <byte_null> <byte_null>
<escaped_symbol_one>   ::= <byte_null> <byte_one>
<escaped_symbol_other> ::= <byte_null> <byte_other>

<chunk_type>  ::= <escaped_symbol_other>
<chunk_start> ::= <chunk_type>

<content_symbol> ::= <unescaped_symbol> | <escaped_symbol_null>
<chunk_content>  ::= <content_symbol>*

<chunk_end> ::= <escaped_symbol_one>

<chunk> ::= <chunk_start> <chunk_content> <chunk_end>
\end{lstlisting}
