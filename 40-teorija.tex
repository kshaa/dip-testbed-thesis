\section{Problēmas pamatnostādne}

Situācija, problēma

Esošie risinājumi īsumā (teikums ar referencēm)

"However..." - ko tie neatrisina vai atrisina nepilnīgi

"Šajā darbā tas tiek atrisināts šādi..." (īsumā, detaļas būs vēlākajās nodaļās)

Standarta paragrāfs: sekojošā nodaļā stāstīts par to, nākamajā par to, ...


Šajā nodaļa mēs apskatam pētāmās problēmas teorētisko pusi.
Veicam daudz $copy + paste + alter \vee trnslate$.
Galvenais ir salikt labi daudz atsauču \cite{kant2018recent}.
Patiesībā katru teikumu varam uztver kā BS un atsauces ir veids kā pateikt, ka tas nav \gls{example}.
\glslink{example}{ekzamplis}.

\section{Saistība ar citiem pētījumiem}

Esošo risinājumu apskats - pieminam kursa darbā rakstīto.

Definē problēmas apgabalu un izmēru un saistību ar citiem līdzīgiem darbiem.  
