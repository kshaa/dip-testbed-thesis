\section{DIP platforma}

Izstrādātā platforma sastāv no \glslink{server}{servera}, \glslink{agent}{aģentiem}
jeb aparatūras starpniekiem, pašas fiziskās attīstītājrīku aparatūras,
\glslink{client}{klientiem} jeb gala lietotājiem ar komandrindas rīkiem, un
pārvaldības paneļa jeb datu pārvaldības tīmekļa lietotnes. Attēlā
\ref{fig:dipdpd0} redzama 0. līm. datu plūsmas diagramma minētai platformai. 

\begin{figure}[H]
    \includegraphics[width=0.7\linewidth]{assets/DPD0.drawio.png}
    \centering
    \caption{DIP platformas 0. līmeņa datu plūsmas diagramma}
    \label{fig:dipdpd0}
\end{figure}

Attēlā \ref{fig:dipdpd0} redzamās apakšsistēmas ir aprakstītas šajā dokumentā
sekojoši. 

Pašā platformas centrā ir centralizēts tīmekļa serveris, kas funkcionē
kā pārvaldības datu avots un platformas lietotāju komunikācijas starpnieks, tā
datu pārvaldības ir aprakstīta nodaļās \ref{sec:usermgmt} un \ref{sec:hwmgmt},
savukārt, tā dalība komunikācijas mehānismos ir aprakstīta \ref{sec:dipactorsystem}.

Nodaļā \ref{sec:dipactorsystem} aprakstītā vājo reāllaika komunikācijas sistēmu,
savukārt, izmanto komandrindas rīki - klients un aģents - lai ļautu lietotājiem
mijiedarboties ar attīstītājrīku aparatūru. Šis mehānisms ir aprakstīts nodaļā
\ref{sec:agentclient}. Un izmantojot šo klienta - aģenta mehānismu, platformā
lietotājam tiek realizētas trīs saskarnes, lai mijiedarbotos ar attīstītājrīku
aparatūru, kuras aprakstītas nodaļās \ref{sec:vinweb}, \ref{sec:vinbytes} un
\ref{sec:vinminos}.

Par šīs platformas, tai skaitā aparatūras laboratoriju, uzstādīšanu un
administrēšanu ir aprakstīts nodaļā \ref{sec:ops}. Par programmaparatūras
izstrādi un testēšanu platformā mazliet aprakstīts nodaļā \ref{sec:usage}. Par
šī darba veikumu un tvērumu aprakstīts nākamajā nodaļā \ref{sec:scope}.

\section{Darba tvērums}
\label{sec:scope}

Darba ietvaros tika izstrādāta vāja reāllaika komunikācijas platforma, kas
nodrošina 1) iespēju reģistrēt un uzskaitīt, pārvaldīt fizisku attīstītājrīku
vai mikrokontrolieru aparatūru, 2) veikt programmaparatūras augšupielādi
aparatūrā, 3) mijiedarboties ar aparatūru, izmantojot vizuālu saskarni, kas
līdzinās noklusētajai aparatūrā fiziski pieejamajai saskarnei, 4) ierobežoti
testēt aparatūras funkcionalitāti. Papildus platforma tika arī testēšanas
nolūkiem izveidota publiskā mākoņpakalpojumu serverī, kas aprakstīts avota koda
repozitorijā un nodaļā \ref{sec:ops}. \cite{VeinbahsKrisjanisTestbed}
\cite{VeinbahsKrisjanisProduction}

Šis darbs izstrādāts atvērti - platformas pirmkods un šī dokumenta pirmkods ir
publiski pieejams GitHub platformā brīvai apskatei un izmantošanai.
\cite{VeinbahsKrisjanisTestbed} \cite{VeinbahsKrisjanisThesis}.

\begin{table}[H]
    \begin{tabular}{ |p{3cm}|p{3cm}|p{3cm}|p{3cm}|p{3cm}| }
    \hline
    Programmēšanas valoda&Faili&Tukšās rindiņas&Komentāru rindiņas&Koda rindiņas\\
    \hline
    Python & 101 & 1551 & 717 & 7424\\
    Scala & 101 & 593 & 120 & 4088\\
    Verilog & 34 & 409 & 676 & 2496\\
    HTML & 11 & 3 & 0 & 492\\
    Bourne Shell & 17 & 91 & 85 & 378\\
    make & 2 & 36 & 30 & 80\\
    SQL & 2 & 33 & 28 & 70\\
    XML & 1 & 14 & 8 & 40\\
    YAML & 2 & 0 & 1 & 33\\
    C++ & 1 & 4 & 14 & 14\\
    \hhline{|=|=|=|=|=|}
    Kopā & 272 & 2734 & 1679 & 15115\\
    \hline
    \end{tabular}
    \centering
    \captionsetup{justification=centering}
    \caption{Koda rindiņu skaita analīze projekta pirmkodā}
    \label{table:cloc}
\end{table}

Aptuvenai sapratnei par izstrādāto koda apjomu, projekta pirmkodā tika izpildīts
koda rindiņu analīzes rīks \cite{AlDanialCloc}, kura rezultāti redzami gan projekta
pirmkoda versiju kontroles repozitorijā \cite{VeinbahsKrisjanisTestbed}, gan tabulā
\ref{table:cloc}.

Platformas pamata funkcionalitāte, lai lietotājs augšupielādētu programmatūru un
veiktu baitu līmeņa datu apmaiņu ar aparatūru, tika realizēta kursa darba laikā.

Bakalaura darba laikā, 1) platformai tika pievienota autentifikācija, 2)
izstrādāts pārvaldības panelis, 3) pārrakstīts klients no tīras grafiskas
saskarnes par termināļa grafisko saskarni, 4) pārrakstīta klienta virtuālā
saskarne kā notikumu sistēma nevis nestrukturizēts kods, 5) izplānota un
realizēta virtuālā saskarne MinOS formālā protokolā, klientā un attīstītājrīkā,
6) tika dokumentēta sistēmas darbība un arhitektūra, tai skaitā arī šī dokumenta
izstrāde.

\section{Statisko datu pārvaldība}
\label{sec:staticdata}

Šī platforma pamatā sastāv no reāllaika datu starpniecības un statisku datu
pārvaldības. Par statiskajiem datiem tiek uzskatīti tādi dati, kurus lietotājs
nemaina izmantojot vājā reāllaika komunikācijas kanālus, kas aprakstīti sadaļā
\ref{sec:dipactorsystem}. 

\begin{figure}[H]
    \includegraphics[width=0.7\linewidth]{assets/physical-er-diagram-gray.png}
    \centering
    \caption{Platformas fiziskā līmeņa datubāzes relāciju diagramma.}
    \label{fig:staticdata}
\end{figure}

Attēlā \ref{fig:staticdata} redzama platformas izstrādes laikā izveidotā
datubāzes fiziskā līmeņa relāciju diagramma, kurā redzamas tabulas lietotājiem
\lstinline!user!, programmaparatūrai \lstinline!software!, aparatūrai
\lstinline!hardware! un aparatūras piekļuvei \lstinline!hardware_access!, kā arī
papildus tabula datubāzes migrāciju vēstures pārvaldībai
\lstinline!play_evolutions!, kas ir noklusēts tabulas nosaukums projektā
izmantotajam Scala programmēšanas valodas tīmekļa lapu izstrādes ietvaram "Play
Framework".

Lielākoties visām biznesa loģikas entītijām ir piesaistīts \gls{uuid}
identifikators \lstinline!uuid! un nosaukums \lstinline!name!, papildus dažām
entītijām ir dažādi piekļuves dati formātā \lstinline!is_*!, programmaparatūrai
ir arī tās saturs \lstinline!content! un lietotājam ir paroles ar jaucējvērtību
datiem \lstinline!hashed_password!.

Programmaparatūras tabulu varētu arī saukt \lstinline!firmware!, taču platformas
izstrādes laikā tika eksperimentēts platformā arī reģistrēt ne tikai
attīstītājrīku aparatūru, bet arī programmējamus mikrokontrolierus, kuros
augšupielādē programmatūru.

Darba ietvaros tika arī apsvērta tabula \lstinline!hardware_messages!, lai
glabātu vēsturi par notikušo vājā reāllaika komunikāciju, taču laika
ierobežojumu dēļ šī funkcionalitāte netika realizēta, jo nebija kritiska
platformas funkcionētspējai.

\section{Lietotāju datu pārvaldība}
\label{sec:usermgmt}

Lietotāju uzskaitei izmantota attēlā \ref{fig:staticdata} redzamā tabula
\lstinline!user!. Lietotājam ir iespējamas trīs iespējamas lomas pārvaldnieks
\lstinline!manager!, laboratorijas īpašnieks \lstinline!lab_owner! un
izstrādātājs \lstinline!developer!, atkarībā no kuras lietotājam platforma
piedēvē attiecīgas tiesības veikt dažādas darbības, kuras skatāmas tabulā
\ref{table:permissions}.

\begin{table}[H]
    \newcounter{permissioncounter}
    \newcommand\rownumber{\stepcounter{permissioncounter}\arabic{permissioncounter}.}
    \begin{tabular}{ |p{1cm}|p{5cm}|p{3cm}|p{6cm}| }
    \hline
    N.p.k.&Darbība&Nepieciešamās tiesības&Papildus nosacījumi \\
    \hline
    \rownumber & Lietotāja bez tiesībām izveide & - & Lietotājvārds nevar būt aizņemts \\
    \hline
    \rownumber & Lietotāju datu uzskaite & \lstinline!is_manager! vai \lstinline!is_lab_owner! & Jebkurš var lasīt savus datus \\
    \hline
    \rownumber & Lietotāja tiesību maiņa & \lstinline!is_manager! & - \\
    \hline
    \rownumber & Programmaparatūras augšupielāde & \lstinline!is_developer! & - \\
    \hline
    \rownumber & Programmaparatūras piekļuve & \lstinline!is_developer! vai
        \lstinline!is_manager! & Jebkurš var lasīt savu vai publicētu
        (\lstinline!is_public!) programmaparatūru \\
    \hline
    \rownumber & Programmaparatūras publicēšana & \lstinline!is_manager! & - \\
    \hline
    \rownumber & Aparatūras izveide & \lstinline!is_lab_owner! & - \\
    \hline
    \rownumber & Aparatūras publicēšana & \lstinline!is_manager! vai
        \lstinline!is_lab_owner! & Pārvaldnieks var publicēt visu, citi tikai
        savu \\
    \hline
    \rownumber & Aparatūras uzskaite & \lstinline!is_manager! vai
        \lstinline!is_developer!, vai \lstinline!is_lab_owner! & Pārvaldnieks
        var uzskaitīt visu, citi tikai savu vai \lstinline!hardware_access!
        tabulā piesaistīto \\
    \hline
    \rownumber & Aparatūras tiesību maiņa & \lstinline!is_manager! vai
        \lstinline!is_developer!, vai \lstinline!is_lab_owner! & Pārvaldnieks
        var pārvaldīt visu, citi tikai savu \\
    \hline
    \end{tabular}
    \centering
    \captionsetup{justification=centering}
    \caption{Lietotāju tiesības un pieejamās darbības}
    \label{table:permissions}
\end{table}

Platformā projekta programmatūras konfigurācijā ir konfigurējams administratora
lietotājs ar visām trīs iepriekšminētajām tiesībām, lai izveidotu sākotnējos
pārvaldības lietotājus.

Platformā lietotājus un to piekļuves datus var konfigurēt no
\glslink{mgmtpanel}{pārvaldības paneļa}, kas ir atšķirīgi no pārējām platformas
darbībām, kas lielākoties norit komandu rindā jeb terminālī, jo šos datus
pārsvarā pārvalda pasniedzējs vai laboratorijas īpašnieks, kuram visvieglāk būtu
no jebkura datora atvērt pārlūku un veikt šīs darbības. Savukārt, izstrādātājam,
kas pārsvarā savu darbu veic komandurindā, lielākā daļa darbības ir pieejamas
izmantojot komandu rindas rīku jeb \glslink{client}{klientu}. Skats no
pārvaldības paneļa lietotāju sadaļas skatāms attēlā \ref{fig:mgmtpanelusr}.

\begin{figure}[H]
    \includegraphics[width=0.9\linewidth]{assets/mgmt-panel-usr-gray.png}
    \centering
    \caption{Platformas pārvaldības paneļa lietotāju pārvaldības skats.}
    \label{fig:mgmtpanelusr}
\end{figure}

\section{Notikumu sistēmu realizācija}
\label{sec:dipactorsystem}

Darbā izstrādātajā platformā, lai veiktu ziņu starpniecību starp lietotāju un
aparatūru, bija nepieciešams realizēt vāja reāllaika komunikāciju. Tā ir
nepieciešama, piemēram, lai sūtītu "LED gaismu" aktuālo stāvokli no aparatūras
lietotājam un "spiedpogu" aktuālo stāvokli no lietotāja aparatūrai jeb lai
realizētu \glslink{vinterface}{virtuālas saskarnes}.

Lai realizētu šādu vāja reāllaika komunikācijas sistēmu, platforma tika pamatā
izstrādāta kā notikumu sistēma, izmantojot Scala programmēšanas valodas aktieru
modeļa programmēšanas ietvaru Akka. Aktieru modelī bāzētas notikumu sistēmas ir
pierādījušas sevi kā spējīgas nodrošināt veiktspēju augstas paralelitātes
reāllaika sistēmām, piemēram, Microsoft ir izstrādājuši aktieru modelī bāzētu
mākoņpakalpojumu risinājumu satvaru Orleans, ar kuru tie spēja realizēt
gājienos-bāzētu spēli "Galactic Reign", kuru tie testēja ar 1000 paralēli
darbojošamies aktieriem, novērojot stabilu veiktspēju ar 95-97\% procesoru
nodarbinātību. \cite[para. 5.1, 5.2]{Bernstein2014}.

Platformas notikumu sistēmas realizācijai pamatā ir nodaļās \ref{sec:actormodel}
un \ref{sec:eventsourcing} pieminētais aktieru modelis un notikumu sistēmas.

Starp \glslink{server}{centralizēto serveri} un \glslink{client}{klientu}, un
\glslink{agent}{aģentu} komunikācija notiek, izmantojot, WebSocket un HTTP
savienojumus. WebSocket savienojumos tiek raidītas gan bināra, gan teksta
formāta ziņas, papildus šādiem savienojumiem tika realizēts, ka pirmai ziņai
vienmēr ir jābūt autentifikācijas ziņai JSON formātā, kas satur lietotājvārdu un
paroli, kā arī tika realizēts, ka ik pēc nokonfigurējama intervāla, kas pēc
noklusējuma bija 30 sekundes, klientam vai aģentam ir jāsūta sirdspuksta (angl.
heartbeat) ziņa, kas, ja netika saņemta 3 reizes pēc kārtas, ļāva secināt, ka
savienojums ir bojāts un to var izbeigt. HTTP savienojumiem autentifikācija tika
realizēta izmantojot HTTP sīkdatnes un HTTP "basic auth" galveni, papildus HTTP
savienojumu gadījumā reāllaika komunikācijas vajadzībām, tika pieņemta
konfigurējama noildze, kas pēc noklusējuma bija 30 sekundes.

Starp \glslink{board}{aparatūru} un \glslink{agent}{aģentu} ir izmantota sadaļā
\ref{sec:serial} aprakstītā seriālā komunikācija. Aģenti, kas realizēti Python
valodā izmantojot daudz dažādas publiski pieejamas bibliotēkas, ik pēc 0.01
sekundes nolasa Linux čaulas seriālo ierīču dziņu buferos ielasītos datus un tos
izsūta kā ziņu WebSocket savienojumā ar serveri. Dati atkarībā no aģenta veida
tiek lasīti pēc dažāda "baudrate", kas pēc noklusējuma ir 115200, kas ir
ātrākais Digilent Anvyl datu pārraides ātrums. Kad aģenta aparatūras datu ziņas
tiek nosūtītas serverī, tās tiek translētas visiem ierīces seriālā porta
abonentiem Akka jeb platformas aktieru klasterī, kas precīzāk aprakstīts sadaļā
\ref{sec:hwmgmt}.

Tabulā \ref{table:serveractors} ir aprakstīti platformas izstrādes gaitā
realizētie serverī esošie aktieri, to funkcija jeb biznesa loģika. Ir vērts
minēt, ka šajā sarakstā nav dažādi Akka vai Play satvaru sistēmas aktieri.

Tabulā \ref{table:cliactors} ir aprakstīti platformas izstrādes gaitā realizētie
aģentā un klientā realizētie aktieri un to apraksts. Šie aktieri ir sarakstīti
kopā, jo reāli tiem pamatā ir tā pati arhitektūra, jo gan klients, gan aģents
abi realizēti vienā Python koda bāzē.

\begin{table}[H]
    \newcounter{serveractorcounter}
    \newcommand\rownumber{\stepcounter{serveractorcounter}\arabic{serveractorcounter}.}
    \begin{tabular}{ |p{1cm}|p{5cm}|p{9cm}| }
    \hline
    N.p.k.&Aktieris&Aktiera funkcija \\
    \hline
    \rownumber & Abonamenti (oriģ. PubSub) & Uzskaita tematus, tā abonentus,
        saņem ziņas, publicē abonentiem \\
    \hline
    \rownumber & Vaicājumi (oriģ. Query) & Palīgaktieris, lai HTTP pieprasījumos
        izveidotu šo aktieri, izsūtītu kādu ziņu, saņemtu atpakaļ ziņu un beigtu
        darbību, kas ir noderīgi, lai integrētu HTTP saskarnes aktieru modelī \\
    \hline
    \rownumber & Tīmekļkamera (oriģ. Camera) & Saņem tīmekļa kameras datus no
        aparatūras un publicē tos tīmekļkameras abonentiem \\
    \hline
    \rownumber & Tīmekļkameras abonents (oriģ. CameraListener) & Klausās
        publicētos tīmekļa kameras datus un pāradresē tos tos tīmekļkameras
        skatītājam HTTP savienojumā kā \lstinline!application/ogg! straumi \\
    \hline
    \rownumber & Aparatūras kontrolieris (oriģ. HardwareControl) & Uztur
        WebSocket savienojumu ar aparatūru, saņem un pāradresē komandas
        aparatūrai, veic seriālās komunikācijas datu starpniecību starp
        aparatūru un aparatūras abonentiem \\
    \hline
    \rownumber & Aparatūras abonents (oriģ. SerialListener) & Uztur WebSocket
        savienojumu ar lietotāju, veic seriālās komunikācijas datu starpniecību
        starp aparatūru un lietotāju \\
    \hline
    \end{tabular}
    \centering
    \captionsetup{justification=centering}
    \caption{Platformas servera aktieri}
    \label{table:serveractors}
\end{table}

\begin{table}[H]
    \newcounter{cliactorcounter}
    \newcommand\rownumber{\stepcounter{cliactorcounter}\arabic{cliactorcounter}.}
    \begin{tabular}{ |p{1cm}|p{5cm}|p{9cm}| }
    \hline
    N.p.k.&Aktieris&Aktiera funkcija \\
    \hline
    \rownumber & Aparatūra (oriģ. Hardware) & Uztur savienojumu ar serveri un
        aparatūru, gaida servera komandas, programmē aparatūru, veic seriālo
        datu komunikāciju starp aparatūru un serveri, ir trīs paveidi
        \lstinline!Anvyl! FPGA attīstīrājrīkam, \lstinline!nrf52!
        mikrokontrolierim un \lstinline!fake! testēšanai \\
    \hline
    \rownumber & Tīmekļkamera (oriģ. Camera) & Uztur savienojumu ar serveri un
        tīmekļkameru, kas fiziski pievienota un pieejama aģenta sistēmā, sūta
        tīmekļkameras datus uz serveri \\
    \hline
    \rownumber & Aparatūras abonents (oriģ. SerialListener) & Uztur WebSocket
        savienojumu ar lietotāju, veic seriālās komunikācijas datu starpniecību
        starp lietotāju un serveri \\
    \hline
    \end{tabular}
    \centering
    \captionsetup{justification=centering}
    \caption{Aģentu un klientu aktieri}
    \label{table:cliactors}
\end{table}

\section{Aparatūras pārvaldība}
\label{sec:hwmgmt}

Iepriekšējās nodaļās ir aprakstīta platformas datu pārvaldība un aktieru
hierarhija, šī nodaļa apraksta kā šī arhitektūra sasaistas kopā, lai realizētu
digitāli pārvaldāmu aparatūru.

Attēlā \ref{fig:labsetup} aprakstīts kā izskatītos platformas sākotnēja
uzstādīšana, lai digitalizētu jeb padarītu pieejamu tiešsaistē laboratorijā
pieejamo aparatūru. Tiek pieņemts, ka katrs servera pieprasījums rezultē
veiksmīgā atbildē.

\begin{figure}[H]
    \includegraphics[width=1.0\linewidth]{assets/lab.png}
    \centering
    \caption{Platformas uzstādīšana laboratorijas digitalizācijai.}
    \label{fig:labsetup}
\end{figure}

Lai sajustu, kāda ir lietotāja pieredze, ir vērts novērtēt aģenta uzstādīšanas
komandu \ref{lst:agentconfig}, kas izveido aktieri \lstinline!Aparatura!,
izveido savienojumu ar serveri un veic aparatūras pārvaldību pēc servera
komandām. Minētā komanda informē komandu rindas rīku, ka aparatūra ir Digilent
Anvyl paveida \lstinline!agent-anvyl!, platformā aparatūra ir reģistrēta ar noteiktu
identifikatoru \lstinline!-b ...!, fiziski piesaistītā aparatūra ir ar noteiktiem
konfigurēšanas/programmēšanas parametriem \lstinline!-n ... -s ...! un ka aparatūrai ir
pieejams noteikts seriālās komunikācijas ports \lstinline!-f ...!.

\begin{lstlisting}[caption={Aģenta uzstādīšana komandu rindā},label={lst:agentconfig},captionpos=b]
dip_client agent-anvyl -b 5b17a393-9004-40ec-a9db-e0bb1e77a0e6 -n Anvyl \
    -s 0 -f /dev/serial/by-id/usb-Digilent_...-if01-port0
\end{lstlisting}

Attēlā \ref{fig:development} aprakstīts kā izskatītos izstrādātāja platformas
izmantošana, lai digitāli un attālināti bez tiešas, fiziskas aparatūras
piekļuves veiktu aparatūras programmēšanu, testēšanu un izmantošanu. Kā arī
komandā \ref{lst:quickrun} ir redzams, ka, lai gan programmēšanas mehānisms
pamatā ir sarežģīts, no izstrādātāja skatpunkta aparatūras attālināta
programmēšana un seriālā komunikācija var tikt abstrahēta gana, lai tas šķistu
diezgan intuitīvi. Komandā redzams norādījums izpildīt programmatūras
augšupielādi serverī un aparatūrā, un uzsākt seriālo komunikāciju
\lstinline!quick-run!, aparatūras identifikators platformā \lstinline!-b ...!,
sakompilētas programmaparatūras faila ceļš izstrādātāja datorā \lstinline!-f ...!
un vēlamās izmantojamās \glslink{vinterface}{virtuālās saskarnes} veids
\lstinline!-t ...!.

\begin{lstlisting}[caption={Programmaparatūras augšupielāde un seriālās komunikācijas klienta komanda},label={lst:quickrun},captionpos=b]
dip_client quick-run -b f65e4276-2d72-4471-b5de-bddbc833d2ea \ 
    -f ./main.bit -t minos
\end{lstlisting}

\begin{figure}[H]
    \includegraphics[width=1.0\linewidth]{assets/agent.png}
    \centering
    \caption{Attīstītājrīka aparatūras attālināta programmēšana.}
    \label{fig:development}
\end{figure}

\section{Tīmekļa kameras virtuālā saskarne}
\label{sec:vinweb}

\begin{figure}[H]
    \includegraphics[width=0.9\linewidth]{assets/mgmt-panel-hw-gray.png}
    \centering
    \caption{Platformas pārvaldības paneļa lietotāju pārvaldības skats.}
    \label{fig:mgmtpanelhw}
\end{figure}

\begin{figure}[H]
    \includegraphics[width=0.9\linewidth]{assets/webcam-usage.png}
    \centering
    \caption{Pārvaldības paneļa aparatūras video straume.}
    \label{fig:hwstream}
\end{figure}

Visi veidi kā es mēģināju realizēt tīmekļkameras virtuālo saskarni.

Cik labi tas strādā?

Kā to varētu izdarīt vēl labāk?

Īsumā - OGG video streaming


\section{Baitu virtuālās saskarnes}
\label{sec:vinbytes}

"hexbytes" un "buttonleds"

5. USB serial - Izmantots komunikācijai starp aģentu un dzelzi


Apraksts dažādu veidu 1-pret-1 baitu ziņapmaiņas virtuālajām saskarnēm.

Ko ar tām var panākt?

Ko ar tām nevar panākt?

\section{MinOS virtuālā saskarne}
\label{sec:vinminos}

MinOS ir virtuāla saskarne, lai mijiedarbotos ar attīstītājrīku aparatūru, kas
ir realizēta kā termināļa grafiskā saskarne, kas redzama attēlā
\ref{fig:minosgui}, taču vizuāli tā līdzinās tai pašai saskarnei, kas pieejama
uz Anvyl attīstītājrīka fiziski, kas redzama attēlā \ref{fig:anvyl}. 

Skatoties attēla \ref{fig:minosgui} saskarnē, ir redzams 8x8 RGB displejs (skat.
kreiso daļu), 8 sarkanas LED gaismas (skat. labās puses 1. rindu), 8 sarkani LED
slēdži (skat. labās puses 2. rindu), 3x8 pelēkas spiedpogas (skat. labās puses
3.-5. rindu), teksta lauki (skat. labās puses 6.-7. rindu). 

\begin{figure}[H]
    \includegraphics[width=0.7\linewidth]{assets/min-os-execution.png}
    \centering
    \caption{MinOS termināļa grafiskā saskarne.}
    \label{fig:minosgui}
\end{figure}

\begin{figure}[H]
    \includegraphics[width=0.7\linewidth]{assets/anvyl.png}
    \centering
    \caption{Anvyl attīstītājrīks.}
    \label{fig:anvyl}
\end{figure}

Lai realizētu MinOS virtuālo saskarni tika izplānots komunikāciju protokols
starp aģentu un klientu. Protokols tika formāli definēts, izstrādājot BNF
formāta sintaksi, kuru iespējams apskatīt pielikumā \ref{att:minosbnf}. Ņemot
vērā, ka vēlamais protokols ir binārs, tad BNF formāts definē baitus kā
heksidecimālu ciparu pārus tātad "0x??".

\ref{lst:minosledchunk}

\begin{lstlisting}[caption={MinOS LED gaismu pakete},label={lst:minosledchunk},captionpos=b]
    0x00 0x02 0xFF 0x00 0x01 
\end{lstlisting}

Pielikums attēls: TUI saskarne https://github.com/kshaa/dip-testbed-dist/blob/master/docs/assets/UHuU1Ur8e0CgoTmsm5khLuOJH.png

Pielikums attēls: 

Ko ar šo saskarni var panākt?

Ko ar šo saskarni nevar panākt?

\section{Infrastruktūras pārvaldība}
\label{sec:ops}

Kā laboratorijas īpašnieki pieslēdz savus dzelžus platformai?

Kā klienti pieslēdzas sistēmai un gūst iespēju izmantot dzelžus?

Kā izstrādātājs jeb darba autors jeb es automatizē versionēšanu, artefaktu pārvaldību?

Docker cross-platform w/ buildx i.e. buildkit

Board management w/ Ansible

CICD deployment procesi

\section{Testēšana un uzturēšana}
\label{sec:usage}

Šo es praktiski neesmu paspējis izdarīt, bet šis būtu interesanti:

Waveform recordings

WASM testing

Ko reāli varētu izdarīt (drīzumā uzkodēšu, šim vajadzētu būt ātri):

MinOS request and response
