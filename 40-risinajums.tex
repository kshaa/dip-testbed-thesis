\section{Motivācija}
Kādēļ es taisīju to, ko es taisīju?  

Kā tas ir labāk par to, kas eksistē tagad?  

\section{Risinājums}

Ko tad es īsti uztaisīju - lielos un mazos vilcienos?
  
No kādām detaļām sastāv mans risinājums?

Šeit mēs aprakstam pāris rindkopās to, kamā mēs izpletīsimies nākamajās nodaļās

Aptuvena ideja: Es uztaisīju aktieru modelī balstītu sistēmu, kas nodrošina centralizētu klientu un aģentu pārvaldību
ar pieņēmumu, ka gan klienti, gan aģenti jebkurā brīdī varētu atslēgties no sistēmas. Aktieru modelis nosaka, ka sistēma
ir arī notikumu sistēma, tātad visas izmaiņas sistēmā tiek padotas apkārt izmantojot ziņas jeb vēstules. Notikumu sistēma
nozīmē, ka ir ļoti viegli klientu un aģentu darbībai paralēli arī žurnalēt to stāvokli piem. datubāzē. Papildus šeit jāpiemin
kādi tad ir tie aktieri, kas tiek izmantoti šajā sistēmā. 

Jāpiemin, ko tad tā sistēma īsti nodrošina - ar Scala Play Framework un HTTP REST un Twirl un Slick tā nodrošina biznesa datu CRUD pārvaldību un
basic auth autentifikāciju, savukārt, ar Scala Play Framework un WebSocket un JSON un aktieriem tā nodrošina vāja reāllaika starpkomunikāciju.  

Vēl jāpiemin, kāda jēga no šīs CRUD pārvaldības, autentifikācijas un vāja reāllaika komunikāciju. CRUD mums ļauj pārvaldīt lietotājus, dzelžus,
programmaparatūru. Vājā reāllaika komunikācija ļauj augšupielādēt mums programmatūru dzelžos, tad sūtīt un saņemt informāciju starp klientu un dzelzi, 
kurā darbinās programmaparatūra, tātad ļauj mums mijiedarboties ar dzelzi. Autentifikācija ļauj pārvaldīt piekļuvi CRUD datiem un
komunikācijai ar dzelžiem.

Vēl vajadzētu pieminēt, neskaitot iepriekšminētās salīdzinoši tehniskās detaļas, kā šī sistēma atrisina biznesa problēmu jeb izstrādes procesa
digitalizāciju. Kā ar šo sistēmu var iesūtīt programmaparatūru, dabūt to dzelzī, mijiedarboties ar programmaparatūru, kas darbinās dzelzī, izmantojot
"virtuālās saskarnes".  

\subsection{DIP platforma}

Šeit varbūt varētu pārkopēt "Piedāvātais risinājums" saturu, lai būtu viena skaidra, īsa, kodolīga sekcija par platformu. 

\subsection{Komunikācijas protokoli}

1. HTTP REST - Biznesa entītiju pārvaldība centrālajā serverī
2. HTTP WebSocket - Vāja reāllaika komunikācija starp centrālo serveri un klientiem/aģentiem
3. Akka - Centrālā servera aktieru vāja reāllaika komunikācija
4. Basic auth un session cookies - autentifikācija iepriekšminētajiem (var pieminēt, var nepieminēt)
5. USB serial - Izmantots komunikācijai starp aģentu un dzelzi

\subsection{Notikumu sistēmas algoritma realizācija}

Notikumu dzinējs ar blakusefektiem - notikumu sistēmas realizācija gan klientā, gan aģentā, gan serverī.

Actor model (every actor defines its inputs)

Message passing.

Event engines.

Event sourcing.

Purity and side-effects.

\subsection{Aģenti un klienti un to starpkomunikācija}

Kā strādā aģenti, kā tie klausās komandas no platformas, kā tie veic programmēšanu,
kā tie uztur virtuālās saskarnes komunikācijas?

Kā strādā klienti? Kā tie sūta CRUD izmaiņas? Kā tie sūta aģentiem komandas? Kā tie
komunicē ar aģentu programmaparatūru izmantojot seriālo portu.

UART, Virtual Interfaces (webcam, serial connection)

\subsection{Virtuālā saskarne "tīmekļkamera"}

Visi veidi kā es mēģināju realizēt tīmekļkameras virtuālo saskarni.

Cik labi tas strādā?

Kā to varētu izdarīt vēl labāk?

Īsumā - OGG video streaming

\subsection{Virtuālās saskarnes "hexbytes" un "buttonleds"}

Apraksts dažādu veidu 1-pret-1 baitu ziņapmaiņas virtuālajām saskarnēm.

Ko ar tām var panākt?

Ko ar tām nevar panākt?

\subsection{Virtuālā saskarne "MinOS"}

Apraksts chunk-to-chunk jeb uz seriālā porta paketēm balstītu ziņapmaiņas saskarni.

Pielikums teksts: Protokola sintakse https://github.com/kshaa/dip-testbed-dist/blob/master/prototypes/06-anvyl-min-os/syntax.bnf  

Pielikums attēls: TUI saskarne https://github.com/kshaa/dip-testbed-dist/blob/master/docs/assets/UHuU1Ur8e0CgoTmsm5khLuOJH.png

Pielikums attēls: 

Ko ar šo saskarni var panākt?

Ko ar šo saskarni nevar panākt?

\subsection{Infrastruktūras pārvaldība}

Kā laboratorijas īpašnieki pieslēdz savus dzelžus platformai?

Kā klienti pieslēdzas sistēmai un gūst iespēju izmantot dzelžus?

Kā izstrādātājs jeb darba autors jeb es automatizē versionēšanu, artefaktu pārvaldību?

Docker cross-platform w/ buildx i.e. buildkit

Board management w/ Ansible

CICD deployment procesi

\subsection{Testēšana un uzturēšana}

Šo es praktiski neesmu paspējis izdarīt, bet šis būtu interesanti:

Waveform recordings

WASM testing

Ko reāli varētu izdarīt (drīzumā uzkodēšu, šim vajadzētu būt ātri):

MinOS request & response
