Digitālās aparatūras projektēšana ir loģisku elementu konfigurāciju un savstarpējo savienojumu plānošana, lai
rezultātā iegūtu jeb projektētu ierīci jeb aparatūru, kas veiktu autora jeb izstrādātāja iedomāto funkcionalitāti, piemēram,
lai izveidotu veļasmašīnas kontrolieri vai datora procesoru, vai citu digitālu risinājumu.  
  
Digitālo aparatūru jeb tās loģisko elementu konfigurāciju var fiziski ražot silikona platēs, tādējādi iegūstot patstāvīgu integrālo
shēmu, vai, alternatīvi, ir iespējams šo loģisko elementu konfigurāciju augšupielādēt attīstītājrīkā, tam nolasot šo konfigurāciju un 
pārkonfigurējot sevī esošu FPGA mikroshēmu, lai tā darbotos atbilstoši dotajai konfigurācijai.  

Darba ietvaros uzmanība tiek koncentrēta uz šī digitālās aparatūras izstrādes dzīvescikla prototipēšanas fāzi jeb brīdi, kad
tiek izstrādāts programmaparatūras prototips, tas tiek augšupielādēts attīstītājrīkā un tiek pārbaudīta attīstītājrīka uzvedība
atbilstoši vēlemajai gala aparatūras funkcionalitātei.

Šī darba mērķis ir mēģināt uzlabot šo izstrādes procesu, pārveidojot attīstītājrīku jeb fiziskās tehnikas pārvaldību, programmēšanu 
un testēšanu no fiziska procesa par digitālu procesu, izstrādājot jaunu platformu šim nolūkam.

Jaunizveidotās platformas pielietojums, sākotnēji ir primāri mērķēts izmantošanai studentiem LU Digitālo Iekārtu Projektēšanas kursā (DIP),
bet reāli koncepts par šādu platformu varētu, iespējams, būt arī plašāk pielietojams profesionālajā industrijā.
  