\section{Problēmas pamatnostādne}

Darbā tiek izstrādāta šāda platforma, tiek apskatītas dažādas tehniskas
problēmas šādas platformas izstrādē un attiecīgi risinājumi šīm problēmām. 

Kursa darbā, tika secināts, ka, lai gan eksistē dažādas tiešsaistes platformas,
lai uzturētu datu apmaiņu starp fizisku aparatūru un platformas lietotājiem, tās
pārsvarā vai nu 1) koncentrējas uz aparatūras izejdatu monitorēšanu vai 2) uz
komandu un atbilžu veida datu apmaiņu.
\cite[para. 3]{VeinbahsKrisjanis2021}

Papildus kursa darba ietvaros tika secināts, ka šo situāciju būtu iespējams
uzlabot izveidojot platformu, kas realizētu vāja reāllaika datu apmaiņu starp
aparatūru un lietotājiem, tādējādi risinot darbam aktuālo problēmu, dodot
lietotājiem iespēju mijiedarboties ar aparatūru attālināti, bet ar vienlīdz
līdzīgu sajūtu kā, ja tas notiktu fiziski, klātienē. \cite[para.
5]{VeinbahsKrisjanis2021} 

Balstoties uz kursa darbā rakstīto, bakalaura darbā tika realizēts sistēmas
pārvaldības panelis, grafiskas termināļa saskarnes baitu virtuālajai saskarnei
un multifunkcionālajai "MinOS" virtuālajai saskarnei, tika izplānota un
izstrādāta multifunkcionāla "MinOS" virtuālā saskarne, tai skaitā protokols un
interaktīvā termināļa saskarne. Papildus platforma tika uzstādīta publiskā
mākonī un platformai tika pieslēgta testa laboratorija ar Digilent Anvyl
attīstītājrīku.

Darbs ir rakstīts kā mērķauditoriju uzskatot cilvēkus ar tehniskām iemaņām,
primāri datorinženierus, taču darbs ir rakstīts tā, lai to varētu saprast
jebkurš datorikas students, pasniedzējs, kolēģis.

Nodaļā \ref{teorija} tiek apskatītas aktieru sistēmas un notikumu sistēmas, lai
saprastu konceptu, kas izmantots, lai realizētu vāja reāllaika komunikāciju
starp \glslink{server}{serveri}, \glslink{agent}{aģentiem} un
\glslink{client}{klientiem}. Savukārt, nodaļā \ref{risinajums} jau tiek
aprakstīta realizētā platforma, tās apakšsistēmas un rīki un darbības mehānismi.

\section{Saistība ar citiem pētījumiem}

Šis darbs ir autora kursa darba konceptuāls turpinājums. Autora kursa darbā tika
veikts virspusējs problēmas apskats par to, kas būtu nepieciešams, lai realizētu
digitālu sistēmu mijiedarbībai starp fizisku digitālu aparatūru un sistēmas
lietotājiem. Šājā darbā tiek izstrādāta konkrēta sistēma, lai panāktu digitālās
aparatūras prototipēšanas fāzes virtualizāciju, kā arī tiek analizēti un
modelēti dažādi procesi un grūtības šādas sistēmas izstrādē.
\cite{VeinbahsKrisjanis2021}

Darbā izstrādātā platforma ir līdzīga citām eksistējošām \gls{fpga}
virtualizācijas platformām, taču pamatā atšķiras ar platformas mērķi un
mērķauditoriju - platforma ir mērķēta attālinātai \gls{fpga} ierīču
programmaparatūras projektēšanai, nodrošinot vāja reāllaika mijiedarbību ar
ierīcēm.

Darba izstrādes brīdī ir pētīti dažādi veidi kā abstrahēt jeb virtualizēt
\gls{fpga} ierīces. \cite[para. I]{VaishnavAnuj2018} Piemēram, "resursu līmenī"
jeb definējot jaunu virtuālu ierīci, kurā projektēt aparatūru ar \gls{hdl},
\cite[para III]{VaishnavAnuj2018}. Piemēram, implementējot virtualizāciju
"mezglu līmenī", kur katrs \gls{fpga} attīstītājrīks ir kā mezlgs, ar ko
mijiedarboties. \cite[para IV]{VaishnavAnuj2018} Kā arī realizējot
virtualizāciju "multi-mezglu līmenī", kas ļauj veikt paralēlus datu apstrādes
darbus vairākās \gls{fpga} ierīcēs. \cite[para V]{VaishnavAnuj2018}

\gls{fpga} virtualizācijai var būt dažādi mērķi, piemēram, lai paātrinātu kādus
datu apstrādes darbus vai lai nodrošinātu piekļuvi grūti pieejamai aparatūrai
(finanšu, ērtību vai fiziskās atrašanās vietas dēļ). Šis darbs koncentrējas uz
attālinātu aparatūras nodrošināšanu studentiem, tādēļ, vispiemērotākais
risinājums, ir tāds, kas būtu noderīgs studentiem un ko būtu viegli atkārtot
jebkurā universitātes laboratorijā. Tādēļ \gls{fpga} ierīču virtualizēšana
abstraktā "resursā" būtu lieka, jo studenti vēlas strādāt ar konkrētu, plaši
dokumentētu aparatūru. "Multi-mezglu" virtualizācija nebūtu gana interaktīva, jo
tā ir paredzēta \textit{ne reāllaika}, grupveida jeb paralēlai datu apstrādei, lai
gan studentiem ir vispirms jāiemācās projektēt vienu ierīci. Tādēļ platformā
izstrādātā "mezgla līmeņa" virtualizācija. 

\gls{fpga} "mezglu līmenī" var panākt lielas datu caurlaidspējas virtualizāciju,
pieslēdzot \gls{fpga} ierīces PCIe ligzdā datoram, kurš funkcionē kā ierīces
pārvaldītājs, taču tas būtu nepamatots finansiāls slogs universitātes
laboratorijai, jo studentiem nav nepieciešama liela datu caurlaidspēja mācību
nolūkos. \cite{WangWei2013} \cite{AsiaticiMikhail2017} 

Tā vietā autors piedāvā sistēmu, kurā komunikācija ar \gls{fpga} ierīcēm notiek
izmantojot seriālo komunikāciju USB savienojumā, kas nodrošina zemāku
veiktspēju, taču padara ierīču uzstādīšanas un uzturēšanas darbus finansiāli
pieejamākus. 

Kā arī daļa šobrīdējās \gls{fpga} ierīču virtualizācijas pētniecības tiek
koncentrēta uz efektīvu \gls{fpga} ierīču izmantošanu jeb, lai nodrošinātu, ka
ierīces nepaliek neizmantotas. \cite[para IV, C]{VaishnavAnuj2018} Taču
laboratorijas un universitātes kursu nolūkos tas ir mazāk aktuāli, jo vienā
semestrī parasti ir fiksēts studentu skaits, kas mācās un izmanto šīs ierīces,
tādēļ izstrādātā platforma ir optimālāka šajā gadījumā, jo tajā netiek risināta
mazāk aktuāla problēma - ierīču daudznomāšana laikā vai telpā.
