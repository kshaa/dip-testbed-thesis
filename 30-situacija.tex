\section{Problēmas pamatnostādne}

Darbā risinātā problēma ir - vai ir iespējams digitalizēt fizisku digitālu
iekārtu projektēšanu, izstrādājot tam paredzētu tiešsaistes platformu?

Darbā tiek izstrādāta šāda platforma, tiek apskatītas dažādas tehniskas
problēmas šādas platformas izstrādē un attiecīgi risinājumi šīm problēmām. 

Šī darba pirmstecī jeb kursa darbā, tika secināts, ka, lai gan eksistē dažādas
tiešsaistes platformas, lai uzturētu datu apmaiņu starp fizisku aparatūru un
platformas lietotājiem, tās pārsvarā vai nu 1) koncentrējas uz aparatūras
izejdatu monitorēšanu vai 2) uz komandu un atbilžu veida datu apmaiņu.
\cite[para. 3]{VeinbahsKrisjanis2021}

Papildus kursa darba ietvaros tika secināts, ka šo situāciju būtu iespējams
uzlabot izveidojot platformu, kas realizētu vāja reāllaika datu apmaiņu starp
aparatūru un lietotājiem, tādējādi risinot darbam aktuālo problēmu, dodot
lietotājiem iespēju mijiedarboties ar aparatūru attālināti, bet ar vienlīdz
līdzīgu sajūtu kā, ja tas notiktu fiziski, klātienē. \cite[para.
5]{VeinbahsKrisjanis2021} 

Darbs ir rakstīts kā mērķauditoriju uzskatot cilvēkus ar tehniskām iemaņām,
primāri datorinženierus, taču darbs ir sarakstīts tā, lai to varētu saprast
jebkurš datorikas students, pasniedzējs, kolēģis.

Nodaļā \ref{teorija} tiek apskatītas aktieru sistēmas un notikumu sistēmas, lai
saprastu konceptu, kas izmantots, lai realizētu vāja reāllaika komunikāciju
starp \glslink{server}{serveri}, \glslink{agent}{aģentiem} un
\glslink{client}{klientiem}. 

\section{Saistība ar citiem pētījumiem}

Šis darbs ir autora kursa darba konceptuāls turpinājums. Autora kursa darbā tika veikts virspusējs problēmas
apskats par to, kas būtu nepieciešams, lai realizētu digitālu sistēmu mijiedarbībai starp fizisku digitālu 
aparatūru un sistēmas lietotājiem. Šājā darbā tiek izstrādāta konkrēta sistēma, lai panāktu digitālās aparatūras 
prototipēšanas fāzes digitalizāciju, kā arī tiek analizēti un modelēti dažādi procesi un grūtības šādas sistēmas
izstrādē. \cite{VeinbahsKrisjanis2021}

