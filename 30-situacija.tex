\section{Problēmas pamatnostādne}

\todo{Recap par prototipēšanas digitalizāciju}

\todo{Atsauce uz kursa darbu par to \& eksistējoši risinājumi nenodrošina soft realtime \& motivācija darbam}

\todo{Paskaidrojums par turpmākajām sadaļām}

Situācija, problēma

Esošie risinājumi īsumā (teikums ar referencēm)

"However..." - ko tie neatrisina vai atrisina nepilnīgi

"Šajā darbā tas tiek atrisināts šādi..." (īsumā, detaļas būs vēlākajās nodaļās)

Standarta paragrāfs: sekojošā nodaļā stāstīts par to, nākamajā par to, ...

\section{Saistība ar citiem pētījumiem}

Šis darbs ir autora kursa darba konceptuāls turpinājums. Autora kursa darbā tika veikts virspusējs problēmas
apskats par to, kas būtu nepieciešams, lai realizētu digitālu sistēmu mijiedarbībai starp fizisku digitālu 
aparatūru un sistēmas lietotājiem. Šājā darbā tiek izstrādāta konkrēta sistēma, lai panāktu digitālās aparatūras 
prototipēšanas fāzes digitalizāciju, kā arī tiek analizēti un modelēti dažādi procesi un grūtības šādas sistēmas
izstrādē. \cite{VeinbahsKrisjanis2021}

