
\documentclass{LU}

\title{Digitālās aparatūras projektēšanas tiešsaistes platforma}	   
\thesistype{Bakalaura darbs}
\author{Krišjānis Veinbahs}
\studentid{kv18042}
\supervisor{prof., Dr. sc. comp. Leo Seļāvo}
\university{Latvijas Universitāte}
\faculty{Datorikas Fakultāte}
\location{Rīga}


% apzīmējumu saraksts
\input{apzimejumi}

\begin{document}

\maketitle

\begin{abstract}
    Īss un kodolīgs darba apraksts!

    Atslēgas vārdi: suns, kaķis, pele.
\end{abstract}
 

\selectlanguage{english}
\begin{abstract}
    Same abstract just in English.

    Keywords: dog, cat, mouse.
\end{abstract}
\selectlanguage{latvian}

\pagenumbering{gobble} 

\tableofcontents

%------------------------------------------------APZĪMĒJUMI---------------------------------------------------------

\printglossary[type=main,title={Apzīmējumu saraksts},toctitle={Apzīmējumu saraksts}]

\pagenumbering{arabic} % sākam numurēt lapas no apzīmējumu saraksta (3. pielikums iekš LU 03.02.2012, 1/38 ) 

\chapter*{Ievads} % * nepieliks numuru pie nosaukuma
\addcontentsline{toc}{chapter}{Ievads}
\pagestyle{plain}
\input{ievads}

%------------------------------------------------DARBS--------------------------------------------------------------

\chapter{Teorija}
\input{teorija}

\chapter{Praktiskā daļa}
\section{Motivācija}	
Motivācija risinājumam?

\section{Analīze}
Actor model (every actor defines its inputs)
Message passing.
Event engines.
Event sourcing.
Purity and side-effects.
Docker cross-platform w/ buildx i.e. buildkit
CICD deployment procesi
OGG video streaming
Waveform recordings
WASM testing
Board management w/ Ansible
UART, Virtual Interfaces (webcam, serial connection)
Improvements using Scala.js and Scala Native (client, agent, web)


%----------------------------------------------SECINĀJUMI----------------------------------------------------------

\chapter*{Secinājumi}
\addcontentsline{toc}{chapter}{Secinājumi}
\input{secinajumi}

%---------------------------------------------LIETERATŪRA----------------------------------------------------------
\renewcommand{\bibname}{Izmantotā literatūra un avoti}
\bibliographystyle{abbrv} % nekur nav minēts kādam jābūt atsaucu noformējumam
\bibliography{main}
\addcontentsline{toc}{chapter}{Izmantotā literatūra un avoti}


%----------------------------------------------PIELIKUMS----------------------------------------------------------

\begin{appendices}
\chapter*{Pielikums}
\input{pielikums}
\end{appendices}

%---------------------------------------------REĢISTRĀCIJAS LAPA (TODO)-------------------------------------------

\chapter*{Reģistrācijas lapa}
\input{registracijaslapa}


\end{document}
