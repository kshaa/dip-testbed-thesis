\makeglossaries

\newglossaryentry{fpga}
{
    name=FPGA,
    description={Lauka programmējamās ventiļu matricas}
}

\newglossaryentry{netlist}
{
    name=Netlist,
    description={Elektronisku elementu savienojumu apraksts jeb elektriskās ķēdes apraksts, ko atgriež HDL valodas}
}

\newglossaryentry{ttl}
{
    name=TTL,
    description={Tranzistora-tranzistora loģika}
}

\newglossaryentry{usb}
{
    name=USB,
    description={Universālā seriālā kopne}
}

\newglossaryentry{uuid}
{
    name=UUID,
    description={Vispārēji unikāls identifikators}
}

\newglossaryentry{uart}
{
    name=UART,
    description={Universālais asinhronais raiduztvērējs}
}

\newglossaryentry{fullduplex}
{
    name=Full duplex,
    description={Pilnduplekss jeb datu pārraide divos virzienos tajā pašā laikā, pretēji "pusduplekss" (angl. "half duplex")}
}

\newglossaryentry{serialport}
{
    name=Serial port,
    description={Seriālā pieslēgvieta jeb pieslēgvieta ar vienu vai diviem slēgumiem, kas nodrošina pilndupleksa 
        vai pusdupleksa datu apmaiņu}
}

\newglossaryentry{server}
{
    name=Platformas serveris,
    description={Izstrādātās platformas sastāvdaļa - serveris, kas nodrošina klientiem un aģentiem savienojumus, 
        lai realizētu centralizētu datu apmaiņu starp tiem}
}

\newglossaryentry{agent}
{
    name=Platformas aģents,
    description={Izstrādātās platformas sastāvdaļa, tipiski mikrokontrolieris, piem. Raspberry Pi, kas 
        funkcionē kā komunikācijas starpnieks starp fiziski pieslēgtu attīstītājrīku un tīklā 
        savienotu centralizētu platformas serveri}
}

\newglossaryentry{pubsub}
{
    name=Pub/Sub,
    description={Asinhrons ziņapmaiņas mehānisms, kur informācijas ražotāji publicē informāciju pēc kāda temata 
        un informācijas klausītāji abonējas šiem tematiem}
}

\newglossaryentry{client}
{
    name=Platformas klients,
    description={Izstrādātās platformas sastāvdaļa - komandrindas rīks, kas nodrošina platformas gala lietotājiem
        iespēju mijiedarboties ar platformas piedāvāto funkcionalitāti}
}

\newglossaryentry{vinterface}
{
    name=Platformas virtuālā saskarne,
    description={Izstrādātās platformas sastāvdaļa - mehānisms jeb tekstuāla vai grafiska saskarne, kas nodrošina 
        platformas lietotājam iespēju mijiedarboties ar attīstītājrīku aparatūru attālināti, imitējot klātienes 
        mijiedarbības pieredzi}
}

\newglossaryentry{mgmtpanel}
{
    name=Platformas pārvaldības panelis,
    description={Izstrādātās platformas sastāvdaļa - grafiska tīmekļa saskarne jeb mājaslapa - pārvaldības panelis, 
        kas nodrošina iespēju autentificēties un apskatīt visus platformā pieejamos statiskos datus, kā arī izveidot
        jaunus lietotājus un pārvaldīt to tiesības}
}

\newglossaryentry{actorsystem}
{
    name=Actor system,
    description={Aktieru sistēma jeb aktieru modeļa koncepts par hierarhisku informācijas sistēmu, kas sastāv no aktieriem}
}

\newglossaryentry{actor}
{
    name=Actor,
    description={Aktieru modeļa koncepts par aktieri jeb informācijas vienību, kam piemīt dzīvescikls, 
        kas var sūtīt ziņas citiem aktieriem, kas var izveidot jaunus aktierus}
}

\newglossaryentry{board}
{
    name=FPGA attīstītājrīka aparatūra,
    description={Izstrādes aparatūra, lai projektētu digitālas iekārtas}
}

\newglossaryentry{petcattle}
{
    name=Pet vs Cattle,
    description={Mājdzīvnieki vs Ganāmpulks. DevOps industrijas koncepts par pārvaldāmas aparatūras veidiem, kur
        par mājdzīvnieku tiktu uzskatīta datubāze, jo tāda parasti ir viena un par to ir ļoti jārūpējas, savukārt,
        par ganāmpulku tiktu uzskatīta slodzes balansējams tīmekļa serveru klasteris, kur jebkura servera instance
        var salūzt, bet klasteris turpinās strādāt slodzes balansēšanas dēļ}
}

\newglossaryentry{wasm}
{
    name=WASM,
    description={WebAssembly. Binārs instrukciju formāts stekā bāzētai virtuālai mašīnai}
}

\newglossaryentry{hdl}
{
    name=HDL,
    description={Aparatūras projektēšanas valoda (angl. Hardware Design Language), piemēram, Verilog un VHDL}
}

\newglossaryentry{dip}
{
    name=DIP,
    description={Digitālo Iekārtu Projektēšana. No Latvijas Universitātes Datorikas fakultātes kursa DatZ7034. 
        Alternatīva nosaukumam "Digitālās Aparatūras Projektēšana", kas izmantots platformas rīkiem}
}