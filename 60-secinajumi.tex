Darba ietvaros ir izstrādāta digitālās \glslink{board}{aparatūras} projektēšanas
tiešsaistes platforma, kas iekļauj dažādus rīkus \glslink{board}{aparatūras}
izstrādātājiem un \glslink{board}{aparatūras} laboratoriju īpašniekiem.
\cite{VeinbahsKrisjanisTestbed}

Precīzāk, ir izstrādāta platforma, kas sastāv no klienta komandu rindas rīka
izstrādātājiem, lai augšupielādētu un testētu platformā
\glslink{firmware}{programmaparatūru}, aģenta komandu rindas rīka laboratorijas
īpašniekiem, lai attālināti pieslēgtu fizisku attīstītājrīku
\glslink{board}{aparatūru} platformai attālinātai programmēšanai un lietošanai.

Papildus platformas ietvaros izstrādāts \glslink{server}{serveris}, kurā
pārvaldīt dažādus lietotāju, \glslink{board}{aparatūras} un
\glslink{firmware}{programmaparatūras} datus un kas funkcionē kā vājas reāllaika
komunikācijas starpnieks starp izstrādātājiem un attālināto
\glslink{board}{aparatūru}.

Papildus platformas ietvaros izstrādātas dažādas virtuālas
\glslink{vinterface}{saskarnes}, lai imitētu fizisku mijiedarbību ar
\glslink{board}{aparatūru} attālinātos apstākļos termināļa vidē. Tai skaitā arī
izstrādāta MinOS grafiskā termināļa virtuālā \glslink{vinterface}{saskarne},
MinOS BNF formāta protokols, MinOS Verilog
\glslink{firmware}{programmaparatūras} modulis, lai abstrahētu attālināto
komunikāciju starp \glslink{board}{aparatūru}, platformu un lietotāju. MinOS
programmaparatūra ir izveidota divās konfigurācijās: "A" ar 5932 \gls{lcs} un
"B" ar 773 \gls{lcs} jeb loģiskajiem elementiem.

Neskaitot visu iepriekšminēto \glslink{firmware}{programmaparatūru} un
\glslink{software}{programmatūru} tika uzstādīta arī publiska testa vide, kurā
lielākoties notika darba izstrāde. Testa vide sastāvēja no
\glslink{server}{servera}, kas atradās virtuālmašīnā publiski pieejama
mākoņpakalpojumu devēja datu centrā, un no laboratorijas, kas atradās darba autora
mājās. \cite{VeinbahsKrisjanisProduction}

Šī darba sākumā tika uzstādīts mērķis, kas pieminēts dokumenta ievadā, mēģināt
uzlabot digitālu iekārtu projektēšanas procesu, pārveidojot fizisko tehnikas
pārvaldību, programmēšanu un testēšanu par digitālu procesu, izstrādājot jaunu
tiešsaistes platformu šim nolūkam.

Šī darbā sākumā paceltā problēma ir - vai ir iespējams virtualizēt fizisku
digitālu iekārtu projektēšanu, izstrādājot tam paredzētu tiešsaistes platformu?
Darba ietvaros šāda platforma ir izstrādāta, tajā tika pievienota
\glslink{board}{aparatūra}, kurā tika attālināti programmēta un testēta
\gls{firmware}, kas ļauj secināt, ka, jā, ir iespējams virtualizēt digitālu
iekārtu projektēšanu, izstrādājot tam paredzētu tiešsaistes platformu.

Darba autors uzskata mērķi par sasniegtu un izpildītu, jo izstrādātajā platformā
ir iespējams reģistrēt un attālināti pārvaldīt fizisku
\glslink{board}{aparatūru} laboratorijas īpašniekam. Ir iespējams šai
\glslink{board}{aparatūrai} dot attālinātu piekļuvi izstrādātājiem.
\glslink{firmware}{Programmaparatūras} izstrādātājiem ir iespējams izstrādāt,
augšupielādēt un testēt \glslink{firmware}{programmaparatūru} attālināti. 

Darba autors uzskata, ka izstrādāto platformu varētu uzlabot, refaktorējot
virtuālās saskarnes no termināļa saskarnēm par grafiskām, piemēram, tīmekļa
pārlūka saskarnēm.

Papildus izstrādāto platformu varētu uzlabot paplašinot to un ļaujot
izstrādātājiem realizēt pašiem savas virtuālās saskarnes.

Kā arī platformas izstrādi varētu paātrināt un atvieglot, rakstot platformas
\glslink{software}{programmatūru} nevis Scala un Python valodās, bet tikai
Scala, lai mazinātu koda duplicēšanos.

Darba autors uzskata, ka darbā apskatītajam tematam ir iespējams arī
turpinājums. Būtu vērtīgi apskatīt šādas platformas pārveidi, lai realizētu
līdzīgu platformu "\gls{fpga} kā pakalpojums" (angl. "\gls{fpga} as a service"),
kurā lietotājiem būtu nevis individuāla piekļuve \glslink{board}{aparatūrai},
bet gan piekļuve ierīču fermai, kas sastāv no vairākām \gls{fpga} ierīcēm. Šādā
platformā varētu, piemēram, veikt paralelizētu datu apstrādi vai testēšanu.
Papildus eksistējošā platforma varētu tikt attīstīta dažādos veidos: 1) seriālās
komunikācijas ierakstīšana un pēcāka lejupielāde un analīze, 2) ierīču
labklājības monitorēšana (angl. health check), 3) automatizēta programmējama
ierīču testēšana, piemēram, ar WASM pirmkodu, kas darbinās aģentā vai serverī,
sūta un saņem signālus, un secina vai \glslink{board}{aparatūras} \gls{firmware}
funkcionē kā paredzēts.
