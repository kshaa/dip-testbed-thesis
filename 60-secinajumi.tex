Darba ietvaros ir izstrādāta digitālās aparatūras projektēšanas tiešsaistes
platforma, kas iekļauj dažādus rīkus aparatūras izstrādātājiem un aparatūras
laboratoriju īpašniekiem.

Precīzāk, ir izstrādāta platforma, kas sastāv no klienta komandu rindas rīka
izstrādātājiem, lai augšupielādētu un testētu platformā programmaparatūru,
aģenta komandu rindas rīka laboratorijas īpašniekiem, lai attālināti pieslēgtu
fizisku attīstītājrīku aparatūru platformai attālinātai programmēšanai un
lietošanai. Papildus platformas ietvaros izstrādāts serveris, kurā pārvaldīt
dažādus lietotāju, aparatūras un programmaparatūras datus un kas funkcionē kā
vājas reāllaika komunikācijas starpnieks starp izstrādātājiem un attālināto
aparatūru. Papildus platformas ietvaros izstrādātas dažādas virtuālas saskarnes,
lai imitētu fizisku mijiedarbību ar aparatūru attālinātos apstākļos termināļa
vidē. Tai skaitā arī izstrādāta MinOS grafiskā termināļa virtuālā saskarne,
MinOS BNF formāta protokols, MinOS Verilog programmaparatūras modulis, lai
abstrahētu attālināto komunikāciju starp aparatūru, platformu un lietotāju.
\cite{VeinbahsKrisjanisTestbed}

Neskaitot visu iepriekšminēto programmaparatūru tika uzstādīta arī publiska
testa vide, kurā lielākoties notika darba izstrāde. Testa vide sastāvēja no
servera, kas atradās virtuālmašīnā publiski pieejama mākoņpakalpojumu devēja
datu centrā, no laboratorijas, kas atradās darba autora mājās.
\cite{VeinbahsKrisjanisProduction}

Šī darba sākumā tika uzstādīts mērķis, kas pieminēts dokumenta ievadā, mēģināt
uzlabot digitālu iekārtu projektēšanas procesu, pārveidojot fizisko tehnikas
pārvaldību, programmēšanu un testēšanu par digitālu procesu, izstrādājot jaunu
tiešsaistes platformu šim nolūkam.

Šī darbā sākumā paceltā problēma ir - vai ir iespējams virtualizēt fizisku
digitālu iekārtu projektēšanu, izstrādājot tam paredzētu tiešsaistes platformu?
Darba ietvaros šāda platforma ir izstrādāta, tajā tika pievienota aparatūra,
kurā tika attālināti programmēta un testēta programmaparatūra, kas ļauj secināt,
ka, jā, ir iespējams virtualizēt digitālu iekārtu projektēšanu, izstrādājot tam
paredzētu tiešsaistes platformu.

Darba autors uzskata mērķi par sasniegtu un izpildītu, jo izstrādātajā platformā
ir iespējams reģistrēt un attālināti pārvaldīt fizisku aparatūru laboratorijas
īpašniekam. Ir iespējams šai aparatūrai dot attālinātu piekļuvi izstrādātājiem.
Programmaparatūras izstrādātājiem ir iespējams izstrādāt, augšupielādēt un
testēt programmaparatūru attālināti. 

Darba autors uzskata, ka darbā apskatītajam tematam ir iespējams arī
turpinājums. Būtu vērtīgi apskatīt šādas platformas pārveidi, lai realizētu
līdzīgu platformu "FPGA kā pakalpojums" (angl. "FPGA as a service"), kurā
lietotājiem būtu nevis individuāla piekļuve aparatūrai, bet gan piekļuve ierīču
fermai, kas sastāv no vairākām FPGA ierīcēm. Šādā platformā varētu, piemēram,
veikt paralelizētu datu apstrādi vai testēšanu. Papildus eksistējošā platforma
varētu tikt attīstīta dažādos veidos: 1) seriālās komunikācijas ierakstīšana un
pēcāka lejupielāde un analīze, 2) ierīču labklājības monitorēšana (angl. health
check), 3) automatizēta programmējama ierīču testēšana, piemēram, ar WASM
pirmkodu, kas darbinās aģentā vai serverī, sūta un saņem signālus, un secina vai
aparatūras programmaparatūra funkcionē kā paredzēts.
