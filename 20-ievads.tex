Digitālās aparatūras projektēšana ir loģisku elementu konfigurāciju un
savstarpējo savienojumu plānošana, lai rezultātā aprakstītu ierīci jeb
aparatūru, kas veiktu autora jeb izstrādātāja iedomāto funkcionalitāti,
piemēram, lai izveidotu veļasmašīnas kontrolieri vai datora procesoru, vai citu
digitālās aparatūras risinājumu.  
  
Digitālās aparatūras prototipēšana ir loģisko elementu konfigurācijas sintēze,
augšupielāde un testēšana. Loģisko elementu konfigurāciju ir iespējams
augšupielādēt \gls{fpga} attīstītājrīkā, tam nolasot šo konfigurāciju un
pārkonfigurējot sevī esošo \gls{fpga} mikroshēmu, lai tā darbotos atbilstoši
dotās aparatūras aprakstam. \cite[para. II]{HerreraAlzu2013}  
  
Darba ietvaros uzmanība tiek koncentrēta uz šī digitālās aparatūras izstrādes
dzīvescikla prototipēšanas fāzi jeb brīdi, kad tiek izstrādāts
\glslink{firmware}{programmaparatūras} prototips, tas tiek augšupielādēts
\glslink{board}{attīstītājrīkā} un tiek pārbaudīta
\glslink{board}{attīstītājrīka} uzvedība atbilstoši vēlamajai gala aparatūras
funkcionalitātei.

Šī darba mērķis ir uzlabot šo izstrādes procesu, pārveidojot attīstītājrīku jeb
fiziskās tehnikas pārvaldību, programmēšanu un testēšanu no fiziska procesa par
digitālu procesu, izstrādājot jaunu platformu šim nolūkam. Darbā risinātā
problēma ir - vai ir iespējams virtualizēt fizisku digitālu iekārtu
projektēšanu, izstrādājot tam paredzētu tiešsaistes platformu?

Eksistē līdzīgas platformas, kas virtualizē jeb abstrahē darbu ar \gls{fpga}
ierīcēm \cite[para. I]{VaishnavAnuj2018}, taču šajā darbā izstrādātā platforma
ir īpaša ar mērķi un mērķauditoriju - attālinātas prototipēšanas darbu
nodrošināsanai studentiem. 

Jaunizveidotās platformas pielietojums, sākotnēji ir mērķēts izmantošanai
studentiem LU Digitālo Iekārtu Projektēšanas (\gls{dip}) kursā (DatZ3074
\cite{DatZ3074}), bet koncepts par šādu platformu būtu arī plašāk pielietojams
profesionālajā industrijā, ja darbā izstrādāto platformu paplašīnātu vairākiem
lietošanas mērķiem.

Darba autoritatīvs nosaukums varētu būt arī "\gls{fpga} kā pakalpojums" (angl.
"\gls{fpga} as a service"), taču šajā darbā \gls{fpga} attīstītājrīki tiek
uzskatīti kā mājdzīvnieki nevis kā ganāmpulks (angl. \gls{petcattle}), tātad
\gls{firmware} tiek augšupielādēta precīzā lietotāja definētā attīstītājrīkā
nevis kādā abstraktā ierīcē, kuru platforma automātiski ieplānotu lietotāja
izmantošanai, kas ir pretēji tipiskām "X as a service" platformām.
